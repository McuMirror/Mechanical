\documentclass[12pt]{article}
\usepackage{tensor}
\usepackage{graphicx}
\usepackage{geometry} % see geometry.pdf on how to lay out the page. There's lots.
\newcommand{\dyad}[1]{\underline{\underline{#1}}}


\begin{document}

\begin{tabular}{@{} rl @{}}
    {\bf To:} & 42 Users \\ 
    {\bf From:} & Eric Stoneking \\ 
    {\bf Date:} &  26 August 2015\\ 
    {\bf Subject:} & Flight Regimes and Orbit Propagation \\ 
\hline
\end{tabular}
  
  
\section{Introduction}

For most applications of 42, orbit motion is the backdrop, not the focus of the study.  But there are some situations where orbit motion takes center stage: proximity ops and operations near minor bodies in particular.  The concept of {\em orbit} has been recently expanded to {\em flight regime} with the addition of infrastructure to support flight regimes from ground rovers and atmospheric flight, to the more familiar orbital environment.  This memo  documents the various flight regimes and methods of propagation used by 42.

It should be emphasized that 42 still employs only fixed-step 4th-order Runge-Kutta numerical integrators.  Yes, one could achieve more numerical accuracy over longer timespans with fewer evaluations by using more sophisticated integrators, but then there would be perpetual fussing over coordinating the longer timesteps that are optimal for orbit propagation and the shorter timesteps required by integration of attitude dynamics.  So far, this extra effort has not been seen to be worthwhile.  42 seeks accuracy by preferring closed-form solutions over numerical propagation where possible, and preferring exact solutions over approximate ones.

\section{Reference Orbits and Regions}

The concept of the {\em reference orbit} is central to 42's architecture.  A reference orbit is an independent point moving under the influence of gravity.  They are defined by the data structure {\tt OrbType} and maintained by the global array of such structures, {\tt Orb}.  One or more (or no) spacecraft may be assigned to a particular reference orbit.  A spacecraft may be fixed to its reference orbit, or it may have relative motion due to initial conditions and perturbation forces.  Relative motion is maintained by the variables {\tt PosR} and {\tt VelR} in the spacecraft ({\tt SC}) data structure.

This architecture is chosen for a couple of reasons.  First, it makes it easy to handle proximity ops or formation flying by assigning all the interacting spacecraft to the same reference orbit.  Second, relative motion can be propagated much more accurately than absolute orbital motion, which is key to simulating precision formation flying.  

For some flight regimes, like ground roving, atmospheric flight, or touching an asteroid, the idea of an orbit becomes awkward.  But the idea of a local reference frame is still very useful.  So we define {\em regions}.  A region is fixed in a World frame.  For the appropriate flight regimes, the region takes the place of a reference orbit.  One or more spacecraft may be assigned to a region.  Motion relative to the region is maintained in the same variables ({\tt PosR, VelR}) that were used for the orbital regime.

\section {Orbit Propagation Methods}

A particle moving under the influence of a massive particle moves in a {\em Keplerian} orbit.  Keplerian orbits may be propagated in (essentially) closed form, and accurately describe a large majority of practical situations of interest.

A spacecraft may be {\em fixed} to its reference orbit.  In this case, perturbing forces are ignored.  Atmospheric drag, solar radiation pressure, even thruster forces may exert torques on the spacecraft, but have no effect on the spacecraft's trajectory.

{\em Encke's method} uses a set of exact nonlinear differential equations to propagate a spacecraft's relative motion under the influence of perturbation forces.  If relative motion is desired, this is the preferred method in most cases.

{\em Euler-Hill} (EH) equations of motion, (aka Clohessy-Wiltshire, etc) are linearized equations of motion for relative motion near a circular reference orbit.  42 provides the capability to propagate EH equations if that is desired, although we typically prefer Encke even in EH-style scenarios.  42 maintains EH states as auxiliary variables at all times --- you don't have to propagate EH to look at a problem with EH concepts. 

{\em Cowell's method} is the simplest and least accurate way to propagate orbital equations of motion.  It is simply integrating $F=ma$ in Cartesian coordinates, with the dominant gravitational forces lumped into $F$ along with all the (typically smaller) perturbation forces.  I don't recommend it for most orbital cases, but it finds a purpose in both three-body problems and near minor bodies.  And for ground ops and atmospheric flight it is the preferred method, since the concept of an unperturbed reference orbit doesn't apply well.

\section{Flight Regimes}

42 identifies four flight regimes: {\em zero}, {\em flight}, {\em central}, and {\em three-body}.  A {\em central} (aka two-body) orbit is the most straightforward and familiar type to spacecraft dynamics.  Circular, elliptical, or hyperbolic, these are the trajectories taken by a particle under the influence of a single point-mass.  Three-body orbits are useful when a spacecraft spends significant periods under the influence of two massive bodies, such as motion near Lagrange points.  {\em Flight} is just motion referred to a Region (as defined above).  The {\em zero} regime is a degenerate case of {\em flight} in which the motion is referred directly to the center of the World.  This is most applicable to motion near a small body.

The value of a reference orbit is to separate the dominant forces (like central point-mass gravitational forces) from relatively small perturbation forces (atmospheric drag, solar radiation pressure, thruster forces, etc).  The solution of the (central orbit) two-body problem is the Keplerian orbit introduced above.  

We extend this idea as best we can to three-body orbital motion.  The few closed-form solutions for three-body motion don't really apply to the real three-body systems we simulate, so the reference orbit is propagated by Cowell's method under the influence of the point-mass gravitational forces of the two main attracting bodies.  Motion of the spacecraft relative to the reference orbit due to perturbation forces may still be neglected (i.e. the spacecraft is fixed to the reference orbit), or may be propagated using Encke's method or Cowell's method.

Consider, for example, a spacecraft starting in a low Earth orbit, then conducting maneuvers to escape Earth, travel to Mars, and insert into a low Mars orbit.  The reference orbit for this mission is initially a central orbit around Earth, and the Sun's and Mars's gravity are small perturbation forces.  The orbit is propagated using closed-form Keplerian motion, and any motion of the spacecraft relative to the reference orbit is probably propagated via Encke's method if at all.  When the spacecraft leaves low orbit, it is on a hyperbolic escape trajectory.  This is still an Earth-centered orbit, until the spacecraft reaches a distance where the Sun's gravitational force is on par with the Earth's pull.  The reference orbit then transitions to become a three-body orbit, in the Sun-Earth gravitational system.  The reference orbit is then propagated using Cowell's method.  Eventually, the spacecraft moves far from Earth, and the reference orbit becomes a central orbit with the Sun as its central body.  Here the reference orbit is a Keplerian orbit again.  As the spacecraft approaches Mars, it again becomes a three-body orbit (in the Sun-Mars gravitational system), then finally a central orbit around Mars.  42 handles all these transitions automatically, and the reference orbit maintains its identity throughout.  The key is that the reference orbit handles the dominant environmental force (which is gravity), and perturbation forces (including third-body gravity) are applied to the spacecraft's relative motion.

This paradigm is challenged by motion near a minor body.  Because the mass of these bodies is so small, solar radiation pressure alone can be nontrivial, and spacecraft-asteroid contact forces are of course commensurate with the gravitational force of the asteroid.  How should we define a reference orbit in these conditions?  In some situations, it still makes sense to use central orbits, if the inaccuracies introduced aren't important to the analysis being performed.  But when landing on an asteroid, the idea of a perturbed Keplerian orbit is stretched to the snapping point.  So we define the so-called {\em zero} flight regime (aka ``zero orbit'') and the {\em flight} flight regime (sorry about the repetition).  

The zero orbit is quite simply a point at rest at the mass center of the attracting body.  Then the relative motion of the spacecraft becomes also its absolute motion.  This motion must be propagated using Cowell's method, as both Encke's method and the Euler-Hill formulation are based on inappropriate assumptions.

The {\em flight} regime is useful for ground roving, launch-and-ascent, atmospheric flight, or entry-descent-and-landing scenarios.  Cowell's method is the method of choice here.

\section{Implementation}

42 requires that each reference orbit be defined in an input file (by convention, with the prefix {\tt Orb\_}) and each spacecraft be defined in its own input file (with the prefix {\tt SC\_}, again by convention).  The orbit type is defined in the orbit input file, and is one of the options {\tt ZERO, FLIGHT, CENTRAL,} or {\tt THREE\_BODY}.  Then information specific to the orbit is supplied in the section corresponding to the orbit type.  The other sections will be ignored by 42.

Also in the orbit definition file is the definition of the {\em formation} frame.  This auxiliary frame is most useful for formation-flying scenarios.  Otherwise, it is typically coincident with the reference orbit.

The spacecraft definition file specifies how the orbital degrees of freedom (OrbDOF) should be propagated.  The spacecraft may be {\tt FIXED} to the reference orbit, or it may wander according to {\tt ENCKE, EULER\_HILL,} or {\tt COWELL}.  Also in the spacecraft input file, the initial position and velocity of the spacecraft with respect to the formation frame is specified.  This is especially useful for prox ops, formation flying, and for zero orbits, ground ops, and flight.  For the first  two, these position and velocity states initialize the variables {\tt PosR} and {\tt VelR}, giving the initial offsets from the reference orbit.  In the zero-orbit case it is emphasized that, since the zero orbit is trivial, the relative offsets are the sum total of the spacecraft position and velocity!

The table below summarizes the orbit types and orbital DOF propagation methods, and which combinations are appropriate.
\begin{table}[h]
\caption{Combinations of Flight Regimes and Orbital DOF Propagation Methods}
\begin{center}
\begin{tabular}{|c||c|c|c|c|} \hline
                           & FIXED & ENCKE & EULER\_HILL & COWELL \\ \hline \hline
 ZERO               &  Not typical & No   & No                & Yes.  Preferred \\ \hline
 FLIGHT            &  Not typical & No   & No                & Yes.  Preferred \\ \hline
 CENTRAL         & Yes      & Yes        & Yes               & Yes, but not preferred \\ \hline
 THREE\_BODY & Yes       & Yes        & No                & Yes, but not preferred \\ \hline

\end{tabular}
\end{center}
\label{tab:Orb}
\end{table}%


\begin{thebibliography}{9}
\bibitem{Battin}
Battin, Richard H., {\it An Introduction to the Mathematics and Methods of Astrodynamics}, AIAA Education Series, 1987.
\bibitem{Roy}
Roy, A. E., {\it Orbital Motion}, Adam Hilger Ltd, Bristol, Great Britain, 1979.
\end{thebibliography}

\end{document}


